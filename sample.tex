\documentclass[UTF8,hyperref]{cumtb}

% 产生 originauth.tex 里的 \Square。
\usepackage{wasysym}
% 提供 verbatiminput 命令和 comment 环境。
\usepackage{verbatim}

% 设置页芯居中。
\geometry{centering}
% 设定行距。
\renewcommand{\baselinestretch}{1.5}

% 使引用标记成为上标。
\newcommand{\supercite}[1]{\textsuperscript{\cite{#1}}}

% 罗列环境中如果每个项目都只有一行左右,则会显得很松散,此时可采用这个命令。
\newcommand{\denseenum}{\setlength{\itemsep}{0pt}}
\hypersetup{colorlinks=true,linkcolor=black}
\begin{document}
	% 各种文档信息。
	\renewcommand{\thesisname}{博士/硕士研究毕业论文}
	% 题目一般不宜超过 20 个字。
	\title{中国矿业大学(北京)论文文档模板}
	\etitle{The CUMTB dissertation document class\\v1.2 beta}
	\author{盖茨波$\cdot$钛$\cdot$维克托}
	\eauthor{Casper Ti.\ Vector}
	\studentid{0XXXXXXX}
	\date{二〇一八年七月}
	\school{理学院}
	\major{统计学}
	\emajor{Statistics}
	\direction{随机 分析}
	\mentor{XX~教授}
	\ementor{Prof.\ XX}
	% 关键词应有 3~5 个。
	\keywords{\LaTeX2e,排版,文档类,\CTeX}
	\ekeywords{\LaTeX2e, typesetting, document class, \CTeX}

	%% 以下为正文之前的部分,页码为小写罗马数字,但不显示页眉和页脚。
	\frontmatter\pagenumbering{roman}\pagestyle{empty}

	\maketitle
	% 版权声明。
	\cleardoublepage
\cleardoublepage
\begin{spacing}{1.5}

\begin{flushleft}
\sihao{中图分类号:  \uline{\makebox[5em]}\hspace{5cm}单位代码:\uline{\quad 11413 \quad}\\
密\quad\qquad  级:  \uline{\makebox[5em]}}
\end{flushleft}
\end{spacing}

\vspace{4em}
\begin{center}
\textbf{\heiti\xiaoyi {硕\quad 士\quad 学\quad 位\quad 论\quad 文}}   %居中,黑体,小一号,加粗
\end{center}
\vspace{2em}
\begin{spacing}{1.5}
\begin{flushleft}
\sihao{中文题目:\uline{\makebox[20em]}}\\
\sihao{英文题目:\uline{\makebox[20em]}}\\
\end{flushleft}
\end{spacing}

\vspace{6em}

\begin{spacing}{1.5}
\begin{flushleft}
\sihao{作\qquad 者:\uline{\makebox[5em]}\hspace{5cm}学\qquad 号:\uline{\makebox[5em]}\\
学科专业:\uline{\makebox[5em]}\hspace{5cm}研究方向:\uline{\makebox[5em]}\\
导\qquad 师:\uline{\makebox[5em]}\hspace{5cm}职\qquad 称:\uline{\makebox[5em]}\\
论文提交日期:\uline{20xx年xx月xx日}论文答辩日期:\uline{20xx年xx月xx日}\\
学位授予日期:\uline{20xx年xx月xx日}}\\
\end{flushleft}
\end{spacing}
\vspace{4em}
\begin{center}
\heiti\xiaoyi {中国矿业大学(北京)}
\end{center}




%
	% 中英文摘要。
	\include{chap/abstract}%
	% 自动生成目录。
	\tableofcontents
	%% 以下为正文,页码为小写罗马数字,但不显示页眉和页脚。
	\mainmatter\pagenumbering{arabic}\pagestyle{fancy}
	% 绪言。
	\specialchap{绪言}

本文档是“中国矿业大学(北京)论文文档模板”的测试和说明文档。

以前的学位论文模板工作由包括~%
dypang\supercite{dypang}、FerretL\supercite{FerretL}、
lwolf\supercite{lwolf}、Langcumtb\supercite{Langcumtb}、
solvethis\supercite{solvethis}~%
的数人做过。
本论文模板是~CUMTB模板的更新版本,
更新的重点是重构和对新文档类、宏包的支持。

CUMTB~文档模板现在的维护者是~Jiandong Wang\footnote%
{\ \href{TSP1600701024@student.cumtb.edu.cn}{\texttt{TSP1600701024@student.cumtb.edu.c}}}。

%
	% 各章节。
	\include{chap/chap1}%
	\include{chap/chap2}
	\include{chap/chap3}
	\include{chap/chap4}
	% 结论。
	\include{chap/conclusion}

	\begin{appendix}
		% 参考文献。
		\bibliographystyle{chinesebst}\bibliography{sample}
		% 此命令手动地在目录中增加相当于章级别的一行。
		\addcontentsline{toc}{chapter}{参考文献}
		% 此命令和真实的一级章命令结合,从而使 \addcontentsline
		% 在目录中产生的页码正常。
		\phantomsection

		% 各附录。
		\include{chap/encl1}
	\end{appendix}

	%% 以下为正文之后的部分,页码为大写罗马数字。
	\backmatter\pagenumbering{Roman}

	% 致谢。
	\chapter{致谢}

感谢 诸位同学的支持,



	% 原创性声明和使用授权说明,不显示页码。
	\pagestyle{empty}
	\include{chap/originauth}
\end{document}

